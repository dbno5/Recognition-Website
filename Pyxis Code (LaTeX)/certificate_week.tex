\documentclass{article}
% page layout
\usepackage[landscape,margin=1in]{geometry}
% line spacing
\usepackage{setspace}
% import and manage images
\usepackage{graphicx}
% fonts
\usepackage{tgschola} % TEX Gyre Schola: qcs
\usepackage{chancery} % URW Chancery: pzc
% input character encoding
\usepackage[utf8]{inputenc}
% European accented characters
\usepackage[T1]{fontenc}
% vertical text alignment
\usepackage{array}
% border (http://tex.stackexchange.com/questions/73611/multiple-border-around-page)
\usepackage{tikz}
% calculate coordinates
\usetikzlibrary{calc}
% curve text
\usetikzlibrary{decorations.text}
% https://en.wikibooks.org/wiki/LaTeX/Macros
% http://tex.stackexchange.com/questions/257128/how-does-the-newcolumntype-command-work
% \newcolumntype{name}[number of arguments]
% {\centering redefines \\ in way that conflicts with tabular and array environments,
%  \arraybackslash restores meaning of \\ for use in tabular and array environments}
% https://en.wikibooks.org/wiki/LaTeX/Tables
% b{'width'} paragraph column with text vertically aligned at the bottom
\newcolumntype{V}[1]{>{\centering\arraybackslash}b{#1}}

\begin{document}
% remove page number
\pagestyle{empty}
% overlay: take picture out of text segment to allow text to be placed on top
% remember picture: save picture size and segments in picture
% current page node refers to rectangular shape that takes up entire page
% http://tex.stackexchange.com/questions/89588/positioning-relative-to-page-in-tikz
% $ (dollar signs) surround expression for inline math mode
% http://www.math.illinois.edu/~ajh/tex/course/intro2.html
\begin{tikzpicture}[overlay,remember picture]
    \draw [teal,double,very thick,rounded corners=16pt]
    ($ (current page.north west) + (0.5cm,-0.5cm) $)
    rectangle
    ($ (current page.south east) + (-0.5cm,0.5cm) $);
    \draw [teal,double,very thick,rounded corners=16pt]
    ($ (current page.north west) + (1cm,-1cm) $)
    rectangle
    ($ (current page.south east) + (-1cm,1cm) $);
\end{tikzpicture}

\begin{center}
    \begin{spacing}{1}
        % \fontsize{size}{baselineskip}
        % baselineskip should be 1.2x size
        \vspace{-20pt}
        \fontfamily{pzc}\fontsize{2cm}{2.4cm}\selectfont
        \begin{tikzpicture}
            % Text along path between two nodes
            % http://tex.stackexchange.com/questions/22314/tikz-bend-text-so-that-it-follows-a-line
            \node (Left) at (-10,0) [] {}; 
            \node (Right) at (10,0) [] {};
            % 
            \path [postaction={decorate,decoration={text along path,text align=center,text={Employee of the Week}}}] (Left) to [bend left=45] (Right);
        \end{tikzpicture}
        \vspace{-100pt}
                        
        {\fontfamily{qcs}\fontsize{1cm}{1.2cm}\selectfont at Pyxis Corporation}
                                        
        {\fontfamily{pzc}\fontsize{1cm}{1.2cm}\selectfont awarded to}
                        
        \vspace{12pt}   
                        
        {\fontfamily{qcs}\fontsize{2cm}{2.4cm}\selectfont *[Recipient Name]*}
    \end{spacing}
\end{center}

% default tabcolsep value is 6pt
\setlength{\tabcolsep}{12pt}
% table is float so h for here
\begin{table}[h]
    \centering
    \begin{tabular} {V{0.20\paperwidth} V{0.20\paperwidth} V{0.20\paperwidth} }       
        % \vspace{0pt} place baselines at bottom
        % b align baselines at bottom and position content at bottom
        % c position content at center
        % \parbox[pos][height][contentpos]{width}{text}
        % \phantom inserts empty box with same dimensions as argument
        \vspace{0pt}\parbox[b][1cm][b]{5cm}{\fontfamily{pzc}\huge\selectfont \centering *[Award Date]* \\
        \rule{5cm}{0.4pt} \\
        \fontfamily{qcs}\normalsize\selectfont \centering Date \\
        \phantom{Date}}                                                             &   
        \vspace{0pt}\parbox[b][1cm][c]{5cm}{\includegraphics[width=5cm]{redribbon}} &   
        \vspace{0pt}\parbox[b][1cm][b]{5cm}{\includegraphics[width=5cm]{rose_mofford_signature} \\
        \rule{5cm}{0.4pt} \\
        \fontfamily{qcs}\normalsize\selectfont \centering *[Authorizing User]* \\
        *[Authorizing User Job Title]*}
    \end{tabular}
\end{table}
\end{document}